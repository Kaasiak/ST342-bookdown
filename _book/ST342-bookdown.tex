% Options for packages loaded elsewhere
\PassOptionsToPackage{unicode}{hyperref}
\PassOptionsToPackage{hyphens}{url}
%
\documentclass[
]{book}
\usepackage{amsmath,amssymb}
\usepackage{lmodern}
\usepackage{iftex}
\ifPDFTeX
  \usepackage[T1]{fontenc}
  \usepackage[utf8]{inputenc}
  \usepackage{textcomp} % provide euro and other symbols
\else % if luatex or xetex
  \usepackage{unicode-math}
  \defaultfontfeatures{Scale=MatchLowercase}
  \defaultfontfeatures[\rmfamily]{Ligatures=TeX,Scale=1}
\fi
% Use upquote if available, for straight quotes in verbatim environments
\IfFileExists{upquote.sty}{\usepackage{upquote}}{}
\IfFileExists{microtype.sty}{% use microtype if available
  \usepackage[]{microtype}
  \UseMicrotypeSet[protrusion]{basicmath} % disable protrusion for tt fonts
}{}
\makeatletter
\@ifundefined{KOMAClassName}{% if non-KOMA class
  \IfFileExists{parskip.sty}{%
    \usepackage{parskip}
  }{% else
    \setlength{\parindent}{0pt}
    \setlength{\parskip}{6pt plus 2pt minus 1pt}}
}{% if KOMA class
  \KOMAoptions{parskip=half}}
\makeatother
\usepackage{xcolor}
\IfFileExists{xurl.sty}{\usepackage{xurl}}{} % add URL line breaks if available
\IfFileExists{bookmark.sty}{\usepackage{bookmark}}{\usepackage{hyperref}}
\hypersetup{
  pdftitle={Probability with Elementary Measure Theory},
  pdfauthor={Leonardo T. Rolla},
  hidelinks,
  pdfcreator={LaTeX via pandoc}}
\urlstyle{same} % disable monospaced font for URLs
\usepackage{longtable,booktabs,array}
\usepackage{calc} % for calculating minipage widths
% Correct order of tables after \paragraph or \subparagraph
\usepackage{etoolbox}
\makeatletter
\patchcmd\longtable{\par}{\if@noskipsec\mbox{}\fi\par}{}{}
\makeatother
% Allow footnotes in longtable head/foot
\IfFileExists{footnotehyper.sty}{\usepackage{footnotehyper}}{\usepackage{footnote}}
\makesavenoteenv{longtable}
\usepackage{graphicx}
\makeatletter
\def\maxwidth{\ifdim\Gin@nat@width>\linewidth\linewidth\else\Gin@nat@width\fi}
\def\maxheight{\ifdim\Gin@nat@height>\textheight\textheight\else\Gin@nat@height\fi}
\makeatother
% Scale images if necessary, so that they will not overflow the page
% margins by default, and it is still possible to overwrite the defaults
% using explicit options in \includegraphics[width, height, ...]{}
\setkeys{Gin}{width=\maxwidth,height=\maxheight,keepaspectratio}
% Set default figure placement to htbp
\makeatletter
\def\fps@figure{htbp}
\makeatother
\setlength{\emergencystretch}{3em} % prevent overfull lines
\providecommand{\tightlist}{%
  \setlength{\itemsep}{0pt}\setlength{\parskip}{0pt}}
\setcounter{secnumdepth}{5}
\usepackage{booktabs}
\usepackage{amsthm}

\usepackage{makeidx}
\makeindex
\usepackage[nottoc]{tocbibind}
\renewcommand{\bibname}{References}

\newtheorem{theorem}[equation]{Theorem}
\newtheorem{proposition}[equation]{Proposition}
\newtheorem{lemma}[equation]{Lemma}
\newtheorem{corollary}[equation]{Corollary}
\theoremstyle{definition}
\newtheorem{definition}[equation]{Definition}
\theoremstyle{remark}
\newtheorem{examplex}[equation]{Example}
\newtheorem{exercisex}[equation]{Exercise}
\newtheorem{remarkx}[equation]{Remark}

\newenvironment{example}{\pushQED{\qed}\renewcommand{\qedsymbol}{\scriptsize$\triangle$}\examplex}{\popQED\endexamplex}
\newenvironment{exercise}{\pushQED{\qed}\renewcommand{\qedsymbol}{\scriptsize$\triangle$}\exercisex}{\popQED\endexercisex}
\newenvironment{remark}{\pushQED{\qed}\renewcommand{\qedsymbol}{\scriptsize$\triangle$}\remarkx}{\popQED\endremarkx}

\newcommand{\dd}{\mathrm{d}}
\newcommand*{\dmu}{\,\dd\mu}
\newcommand{\s}{\ensuremath{\sigma}}
\newcommand{\sE}{\sigma(\cE)}
\newcommand{\into}{\int_\Omega}
\newcommand{\oR}{\overline{\R}}

\let\mathbb\relax % remove the definition by unicode-math
\DeclareMathAlphabet{\mathbb}{U}{msb}{m}{n}

% redefine missing \setminus in XeLaTex as \backslash
\AtBeginDocument{\renewcommand{\setminus}{\mathbin{\backslash}}}

\newcommand{\CC}{\mathbb{C}}
\newcommand{\E}{\mathbb{E}}
\newcommand{\N}{\mathbb{N}}
\newcommand{\Pb}{\mathbb{P}}
\newcommand{\Q}{\mathbb{Q}}
\newcommand{\R}{\mathbb{R}}
\newcommand{\SSS}{\mathbb{S}}
\newcommand{\V}{\mathbb{V}}
\newcommand{\Z}{\mathbb{Z}}

\newcommand{\cA}{\mathcal{A}}
\newcommand{\cB}{\mathcal{B}}
\newcommand{\cC}{\mathcal{C}}
\newcommand{\cD}{\mathcal{D}}
\newcommand{\cE}{\mathcal{E}}
\newcommand{\cF}{\mathcal{F}}
\newcommand{\cG}{\mathcal{G}}
\newcommand{\cL}{\mathcal{L}}
\newcommand{\cP}{\mathcal{P}}

\newcommand{\I}{\mathds{1}}

\newcommand{\ptimes}{\otimes}
\newcommand{\mtimes}{\otimes}

\renewcommand*{\leq}{\leqslant}
\renewcommand*{\geq}{\geqslant}


\allowdisplaybreaks
\hyphenpenalty 9900
\tolerance 5000
\hypersetup{hidelinks,colorlinks,urlcolor=blue,linkcolor=[rgb]{.5,0,0},citecolor=[rgb]{.5,0,0}}


\makeatletter
\def\thm@space@setup{%
  \thm@preskip=8pt plus 2pt minus 4pt
  \thm@postskip=\thm@preskip
}
\makeatother
\ifLuaTeX
  \usepackage{selnolig}  % disable illegal ligatures
\fi
\usepackage[]{natbib}
\bibliographystyle{apalike}

\title{Probability with Elementary Measure Theory}
\author{Leonardo T. Rolla}
\date{2021-08-09}

\begin{document}
\maketitle

{
\setcounter{tocdepth}{1}
\tableofcontents
}
\hypertarget{preface}{%
\chapter*{Preface}\label{preface}}
\addcontentsline{toc}{chapter}{Preface}

\textbf{Reusing this material}

This licence allows users to distribute, remix, adapt, and build upon
the material in any medium or format, so long as attribution is given to
the creator. The licence allows for commercial use. If you remix, adapt,
or build upon the material, you must license the modified version under
identical terms.

\textbf{Source code}

If you make changes to these notes, please distribute the modified
source code along with the typeset material, and please keep this
request to those who receive it. This is a kind request from the author,
not a legal requirement.

\textbf{Attribution}

This is an updated and expanded version of the lecture notes
``Mathematics of Random Events'' written in collaboration with Nikolaos
Constantinou in 2020. Those notes were largely based on material typeset
by Nikolaos Constantinou as a student, when the ST342 module was taught
by Wei Wu in the Autumn 2019 term at Warwick. The module taught by Wei
Wu was largely based on handwritten lecture notes received from Larbi
Alili, which in turn were based on material previously developed by
Sigurd Assing, Anastasia Papavasileiou, Jon Warren and Wilfrid Kendall.

\hypertarget{sec:intro}{%
\chapter{Infinity}\label{sec:intro}}

Can we toss a coin infinitely many times?

Why is countable additivity important?

We know that \(\E[X+Y]=\E[X] + \E[Y]\), but why? It doesn't seem to follow
from the usual definition that treats discrete and continuous random
variables as hermetically separated entities. So, what is expectation,
really?

Do continuous random variables even exist?

How small can a class \(\cA\) of events be so that their probabilities
determine \(\Pb\)?

Why can we differentiate moment generating functions to compute moments?

Our aim is to study concepts of Measure Theory useful to Probability
Theory, providing a solid ground for the latter. A \emph{measure} generalises
the notion of area for arbitrary sets in Euclidean spaces
\(\mathbb{R}^d\), \(d \geq 1\). We introduce the theory of measurable
spaces, measurable functions, integral with respect to a measure,
density of measures, product measures, and convergence of functions and
random variables.

Below we give examples that motivate the need for such a theory, discuss
in which sense modern Measure Theory is the best we can hope for, and
introduce the concept of infinite numbers and infinite sums used
throughout the remaining chapters.

\hypertarget{sub:eventsinfty}{%
\section{Events at infinity}\label{sub:eventsinfty}}

We know that \(A_n \uparrow A\) implies \(\Pb(A_n) \uparrow \Pb(A)\) which,
assuming that \(\Pb\) is non-negative and finitely additive, is equivalent
to \(\Pb\) being \(\sigma\)-additive (a shorthand for countably additive).
Below we see examples where interesting models and events require some
sort of limiting process in their study.

\begin{example}[Ruin Probability]
Gambling with initial wealth \(X_0 \in \mathbb{N}_0\). For any \(t \geq 1\),
we bet an integer amount and reach a wealth denoted by \(X_t\). If at any
point in time, wealth amounts to \(0\), it remains \(0\) forever. The sample
space that indicates the wealth process is
\(\Omega = \{(x_0, x_1, \dots): x_i \in \N_0 \}\). We define the function
of wealth after the \(n\)-th gamble by,
\(X_n : \Omega \to \mathbb{N}\cup\{0\}\), where \(X_n(\omega) = x_n\). In
fact, \((X_n)_{n \geq 0}\) is a Markov process. Then
\[\{ \text{stay in state $ 0 $ eventually} \}
= \bigcup_{n=0}^{\infty}\bigcap_{m = n}^{\infty}\{\omega \in \Omega: X_m(\omega) = 0\}.\]
By monotonicity and continuity, its probability is \[\Pb(
\{ \text{stay in state $ 0 $ eventually} \}
)
=
\lim_{n\to\infty}
\Pb(\{\omega \in \Omega: X_n(\omega) = 0\})
 .\]
\end{example}

\begin{example}[Brownian motion]
It is possible to construct a sequence of continuous piece-wise linear
functions, which in the limit give a continuous nowhere differentiable
random path that is at the core of Stochastic Analysis.
\end{example}

\begin{example}[Uniform variable]
If \(X \sim U[0,1]\), then \(\Pb(X \ne x) = 1\) for every \(x \in \R\).
Nevertheless, \(\Pb(X \ne x \text{ for every } x \in \R ) = 0\), so there
will be some unlucky \(x\) that will happen to be hit by \(X\). Now for a
countable set \(A\), by \(\sigma\)-additivity we have
\(\Pb(X \in A) = \sum_{x \in A} \Pb(X=x) = 0\). Since \(\Q\) is countable,
we see that a uniform random variable is always irrational. Well, this
is unless there is an uncountable number of such variables in the same
probability space, in which case some unlucky variables may happen to
take rational values.
\end{example}

\begin{example}
Let \(X_1,X_2,X_3,\dots\) be independent and take value \(\pm 1\) with
probability \(\frac{1}{2}\) each, and take \(S_n = X_1 + \dots + X_n\). The
strong law of large numbers says that, almost surely (a shorthand for
``\(\Pb(\cdots)=1\)''), \(\frac{S_n}{n} \to 0\) as \(n \to \infty\). In order to
show this, we need to express this event as a limit and compute its
probability. The first part is simply:
\[\{\lim_{n \to \infty} \tfrac{S_n}{n} = 0\}
=
%\{\forall k, \exists n_0, \forall n \geq n_0, |\tfrac{S_n}{n}| < \tfrac{1}{k} \}
%=
%\\
%=
\bigcap_{k=1}^\infty
\bigcup_{n_0=1}^\infty
\bigcap_{n=n_0}^\infty
\{|\tfrac{S_n}{n}| < \tfrac{1}{k} \}
.\]
\end{example}

\begin{example}[Recurrence of a random walk]
For the sequence \((S_n)_n\) of the previous example, we know that
\(\frac{S_n}{n} \to 0\) with probability one. We want to consider whether
\(\frac{S_n}{n}\) converges to zero from above, from below, or
oscillating, which is the same as asking whether \(S_n=0\) infinitely
often.
\end{example}

\hypertarget{sub:measureproblem}{%
\section{The measure problem}\label{sub:measureproblem}}

It is clear (it should be!) from previous examples that we want to work
with measures that have nice \emph{continuity} properties, so we can take
limits. However, when the mass is spread over uncountably many sample
points \(\omega \in \Omega\), it is not possible to assign a measure to
all subsets of \(\Omega\) in a reasonable way.

We would like to define a random variable uniformly distributed on
\([0,1]\), by means of a function that assigns a weight to subsets of this
interval. For instance, what is the probability that this number is
irrational? What is the probability that its decimal expansion does not
contain a 3?

This is the same problem as assigning a `length' to subsets of \(\R\). We
are also interested in defining a measure of `area' on \(\R^2\), `volume'
on \(\R^3\), and so on.

Of course, a good measure of length/area/volume/etc. on \(\R^d\) should:

\begin{enumerate}
\def\labelenumi{\arabic{enumi}.}
\item
  give the correct value on obvious sets, such as intervals and balls;
\item
  give the same value if we rotate or translate a set;
\item
  be \(\sigma\)-additive.
\end{enumerate}

We stress again that \(\sigma\)-additivity is equivalent to a measure being
continuous, and we are not willing to resign that. On the other hand, we
do not want more than that: each of the uncountably many points in
\([0,1]\) alone has length zero, but all together they have length one;
likewise, each sequence of coin tosses in \(\{0,1\}^{\N}\) has probability
zero, but all together they have probability one.

The measure problem is the following.

\begin{quote}
There is no measure \(m\) defined on all subsets of \(\R^d\) which satisfy
all the reasonable properties listed above. What modern Measure Theory
does is to work with measures that are defined on a class of sets which
is large enough to be useful and small enough for these properties to
hold.
\end{quote}

The next example shows that there is no measure \(m\) defined on all
subsets of \(\R^3\) which satisfies these three properties.

\begin{example}[Banach-Tarski paradox]
\protect\hypertarget{exm:banachtarski}{}\label{exm:banachtarski}Consider the ball \[B = \{x \in \R^3 : \|x\| \leq 1 \} .\] There exist\footnote{See \url{https://youtu.be/s86-Z-CbaHA} for a nice overview of the
  proof.}
\(k\in\N\), \emph{disjoint} sets \(A_1,\dots,A_{2k}\), and \emph{isometries} (maps
that preserve distances and angles) \(S_1,\dots,S_{2k}:\R^3\to\R^3\) such
that
\[B = (A_1 \cup \dots \cup A_k) \cup (A_{k+1} \cup \dots \cup A_{2k}),\]
\[B = S_1 A_1 \cup \dots \cup S_{k} A_{k}, \quad B = S_{k+1} A_{k+1} \cup \dots \cup S_{2k} A_{2k}.\]
So \(B\) was decomposed into \emph{finitely many} pieces, which were later on
moved around \emph{rigidly} and \emph{recombined} to produce two copies of \(B\)!
Why is it a paradox? Finitely many pieces is not the issue in itself,
since \(\N\) can be decomposed into even and odd numbers, and they can be
\emph{compressed} (or stretched, in some sense) to produce two copies of
\(\N\). Rigidity alone is not the issue either, since we can move each of
the \emph{uncountably many} points of the segment \([0,1]\) to form the segment
\([0,2]\). The paradox is that this magic was done with \emph{rigid movements
on finitely many pieces}. And here we can see the \emph{measure problem}: if
all these disjoint sets \(A_1,\dots,A_{2k}\) were to have a volume
\(V_1,\dots,V_{2k} \geq 0\), what would be the volume of the ball \(B\)?
\end{example}

The next example is not nearly as effective in impressing friends at a
party, and would certainly not make a youtube video with 31 million
views, but it has two advantages. First, it shows directly that the
measure problem already occurs on \(d=1\). Second, we can actually explain
its proof in a third of a page rather than a dozen.

\begin{example}[Vitali Set]
\protect\hypertarget{exm:vitali}{}\label{exm:vitali}Consider the unit circle \(\SSS^1\) with points indexed by turns instead of degrees or radians.
This is the same as the interval \(\SSS^1 = [0,1)\) with the angle
addition operation \(x \oplus y = x+y \mod 1\). There exists a set
\(E \subseteq [0,1)\) such that \(\SSS^1\) is decomposed into disjoint
\(\{E_n\}_{n\in\N}\) which are translations of \(E\). And here we see again
the \emph{measure problem}: by \(\sigma\)-additivity, if the length of \(E\) is
zero, then the length of the circle is zero; and if the length of \(E\) is
non-zero, then the length of the circle is infinite. So \(E\) is not
measurable.
\end{example}

\begin{proof}
\emph{(Sketch)} Write \(\Q \cap [0,1) = \{r_n\}_{n=1,2,3,\dots}\). For
\(E \subseteq \SSS^1\), let \(E_n = \{ x \oplus r_n : x \in E \}\) be the
\emph{translation} of \(E\) by \(r_n\). We want to find a set \(E\) such that

\begin{enumerate}
\def\labelenumi{\arabic{enumi}.}
\item
  The sets \(E_1,E_2,E_3,\dots\) are disjoint,
\item
  The union satisfies \(\cup_n E_n = \SSS^1\).
\end{enumerate}

Start with a small set that satisfies the first property, such as
\(E = \{0\}\). Enlarge the set \(E\) by adding a point
\(x \in \SSS^1 \setminus (\cup_n E_n)\). Adding such point does not break
the first property (proof omitted), and may help the second one. Keep
adding points this way, until it is no longer possible. When it is no
longer possible, it can only be so because the second property is also
satisfied.\footnote{See \citep[ 1.4.9]{Cohn13} for a complete proof.}
\end{proof}

\begin{remark}
It is often emphasised that the Banach-Tarski paradox and Vitali set
depend crucially on the Axiom of Choice (for the above sketch of proof,
it is concealed in the expression ``keep adding until''). We may wonder
what happens if we do not accept this axiom. In this case, we cannot
prove the Banach-Tarski paradox, nor the existence of a Vitali set. But
neither can we prove that they do not exist, so the measure problem
persists.
\end{remark}

\hypertarget{sub:infinitenumbers}{%
\section{Infinite numbers and infinite sums}\label{sub:infinitenumbers}}

We now define the set of extended real numbers and briefly discuss some
its useful properties, then discuss the meaning of infinite sums, and
move on to other perhaps philosophical questions about this theory.

\hypertarget{subsub:extended}{%
\subsection{Extended real numbers}\label{subsub:extended}}

We are about to start working with measures, and because measures can be
infinite, and integrals can be negative infinite, we work with the set
of \emph{extended real numbers} \(\oR := [-\infty,+\infty]\) that extends \(\R\)
by adding two symbols \(\pm \infty\). The novelty is of course to
conveniently allow operations and comparisons involving these symbols.

Basically, we can safely operate as one would reasonably guess:
\[\begin{gathered}
-\infty < -1 < 0 < 5 < +\infty,
\
-7 +\infty = +\infty,
\
(-2) \times (-\infty) = +\infty,
\\
|-\infty| = +\infty,
\
(+\infty) \times (-\infty) = -\infty,
\
a \leq b \implies a+x \leq b+x
\\
\lim_{n\to\infty} (2+n^2)
=
\lim_{n\to\infty} 2
+
\lim_{n\to\infty} n^2
=
2 + \infty
=
+\infty,\end{gathered}\] etc. Since we will never need to divide by
infinity, let us leave \(\frac{x}{\infty}\) undefined (otherwise we would
need to check that \(x\) is finite).

The non-obvious definition is \(0 \cdot \infty = 0\). In Calculus, it
would have been considered an indeterminate form, but in Measure Theory
it is convenient to define it this way because the integral of a
function that takes value \(0\) on an interval of infinite length and
\(+\infty\) at a few points should still be \(0\). That is, the area of a
rectangle having zero width and infinite length is zero.

Now some caveats. First, \(\lim_n (a_n b_n) = (\lim_n a_n)(\lim_n b_n)\)
may fail in case it gives \(0 \cdot \infty\). Also, note that now
\(a+b = a+c\) does not imply \(b=c\). This can be false when \(a=\pm \infty\).
Likewise, \(a<b\) no longer implies that \(a+x < b+x\). So we should be
careful with cancellations.

\begin{quote}
The one thing that is definitely not allowed, and that Measure Theory
does not handle well, is \[\text{" $ +\infty -\infty $ " !}\] This is
simply forbidden, and if we will ever write this, it will be in
quotation marks and just in order to say that this case is excluded.
\end{quote}

The reader should consult \citep[ §§B.4--B.6]{Cohn13} and \citep[ p.~xi]{Tao11} for a
more complete description of operations on \(\oR\).

\hypertarget{infinite-sums}{%
\subsection{Infinite sums}\label{infinite-sums}}

Infinite sums of numbers on \([0,+\infty]\) are always well-defined
through a rather simple formula. If \(\Lambda\) is an index set and
\(x_\alpha \in [0,+\infty]\) for all \(\alpha \in \Lambda\), we define:

\[\nonumber
\sum_{\alpha \in \Lambda} x_\alpha = \sup_{\stackrel{A \subseteq \Lambda}{A \text{ finite}}} \sum_{\alpha \in A} x_\alpha
.\] The set \(\Lambda\) can be uncountable, but the sum can be finite only
if \(\Lambda_+ = \{\alpha:x_\alpha>0\}\) is countable (proof omitted). If
\(x_\alpha \in [-\infty,+\infty]\), we define
\(\Lambda_- = \{\alpha:x_\alpha<0\}\) and
\begin{equation}
\label{eq:infsum}
\sum_{\alpha \in \Lambda} x_\alpha =
\sup_{\stackrel{A \subseteq \Lambda_+}{A \text{ finite}}} \sum_{\alpha \in A} x_\alpha
-
\sup_{\stackrel{A \subseteq \Lambda_-}{A \text{ finite}}} \sum_{\alpha \in A} -x_\alpha
,
\end{equation}

as long as this difference does not give ``\(+\infty-\infty\)'' !

The theory of conditionally convergent sums as
\begin{equation}
\sum_{j \in \N} x_j
=
\lim_n
\sum_{j = 1}^n x_j
\label{eq:series}
\end{equation}
is hardly meaningful to us. In case the expression
in \eqref{eq:infsum} is well-defined, we can write
\((\Lambda_- \cup \Lambda_+) = \{\alpha_j\}_{j\in\N}\) by ordering these
indices in any way we want (assuming for simplicity that these sets are
countable), and formula \eqref{eq:series} will give the same result
as \eqref{eq:infsum}. Pretty robust.

However, in case \eqref{eq:series} converges
but \eqref{eq:infsum} is not well-defined (so it gives
``\(+\infty-\infty\)''), we can re-order the index set \(\N\) so
that \eqref{eq:series} will give any number we want. This is definitely
not the type of delicacy we want to handle here.

For this reason, we will only
use \eqref{eq:series} when either \[x_j \in [0,+\infty]\] for all \(j\),
or when \[\sum_j |x_j| < \infty .\] So there are two overlapping cases
where we can work comfortably: non-negative extended numbers, or series
which are absolutely summable.

\hypertarget{two-different-and-overlapping-theories}{%
\subsection{Two different and overlapping theories}\label{two-different-and-overlapping-theories}}

The above tradeoff is already a good prelude to something rather deep
that will appear constantly in upcoming chapters. Borrowing
from \citep{Tao11}:

\begin{quote}
Because of this tradeoff, we will see two overlapping types of measure
and integration theory: the \emph{non-negative} theory, which involves
quantities taking values in \([0, +\infty]\), and the \emph{absolutely
integrable} theory, which involves quantities taking values in \(\R\) or
\(\CC\).
\end{quote}

However, at the risk of leaving it for the reader to figure out some
corner cases, we can (and will) extend these theories to a theory on
\([-\infty,+\infty]\) by doing what we just did above. Namely, whereas the
absolutely integrable theory requires that both terms
in \eqref{eq:infsum} be finite, and the non-negative theory requires
that one of them be zero, we only require that one of them be finite.

We end this chapter with the simplest example of these overlapping
theories

\begin{theorem}[Tonelli Theorem for series]
\protect\hypertarget{thm:seriestonelli}{}\label{thm:seriestonelli}Let \(x_{m,n} \in [0,+\infty]\) be a doubly-indexed sequence. Then
\[\sum_{m=1}^\infty
\sum_{n=1}^\infty
x_{m,n}
=
\sum_{(m,n)\in\N^2} x_{m,n}
=
\sum_{n=1}^\infty
\sum_{m=1}^\infty
x_{m,n}
.\]
\end{theorem}

\begin{proof}
A proof is given in §\protect\hyperlink{sub:fubini}{7.2} as an application of Tonelli Theorem, but here
is a bare hands proof. For \(A \subseteq \N^2\) finite, there exist
\(m,n\in\N\) such that
\(\sum_{A}x_{j,k} \leq \sum_{j=1}^n \sum_{k=1}^n x_{j,k}\). Conversely,
given \(m,n\in\N\), there is \(A \subseteq \N^2\) such that
\(\sum_{A} x_{j,k} = \sum_{j=1}^n \sum_{k=1}^n x_{j,k}\). Therefore,
\[\begin{gathered}
\sum_{(j,k)\in\N^2} x_{j,k}
=
\sup_A \sum_{(j,k)\in A} x_{j,k}
=
\sup_{m,n\in\N} \sum_{j=1}^m \sum_{k=1}^n x_{j,k}
=
\\
=
\sup_{m\in\N}
\sup_{n\in\N}
\sum_{j=1}^m \sum_{k=1}^n x_{j,k}
=
\sup_{m\in\N} \sum_{j=1}^m
\Big(
\sup_{n\in\N} \sum_{k=1}^n
x_{j,k}
\Big)
=
\sum_{j=1}^\infty
\Big(
\sum_{k=1}^\infty
x_{j,k}
\Big)
%
.\end{gathered}\] The other equality is proved in identical way.
\end{proof}

\begin{theorem}[Fubini Series]
\protect\hypertarget{thm:seriesfubini}{}\label{thm:seriesfubini}Let
\(x_{m,n} \in [-\infty,+\infty]\) be a doubly-indexed sequence. If
\[\sum_{m=1}^\infty
\sum_{n=1}^\infty
|x_{m,n}|
<\infty
,\] then \[\sum_{m=1}^\infty
\sum_{n=1}^\infty
x_{m,n}
=
\sum_{(m,n)\in\N^2} x_{m,n}
=
\sum_{n=1}^\infty
\sum_{m=1}^\infty
x_{m,n}
.\]
\end{theorem}

\begin{proof}
Given in §\protect\hyperlink{sub:fubini}{7.2} as an application of Fubini Theorem.
\end{proof}

\begin{example}
Consider the doubly-indexed sequence \[x_{m,n}=
\begin{array}{|rrrrrr}
\hline
1 & -1 & 0 & 0 & 0 & \dots
\\
0 & 1 & -1 & 0 & 0 & \dots
\\
0 & 0 & 1 & -1 & 0 & \dots
\\
0 & 0 & 0 & 1 & -1 & \dots
\\
\vdots &
\vdots &
\vdots &
\vdots &
\ddots &
\ddots
\end{array}\] which is not absolutely summable. Note that summing
columns and then rows we get \(1\), whereas summing rows and then columns
we get \(0\).
\end{example}

\hypertarget{sub:dummy}{%
\section{Dummy subsection}\label{sub:dummy}}

This is a dummy equation

This section is to show how we cen refer to other sections and theorems.

Referencing the Theorems, exmalpes and equations is done by using their labels
It is enough to enter \texttt{\textbackslash{}@ref(label)} e.g.:

\begin{itemize}
\tightlist
\item
  Theorem \ref{thm:seriesfubini} is a very important one.
\item
  Don't forget to solve Example \ref{exm:vitali} before the exam.
\item
  Equation \eqref{eq:series} can be quite tricky.
\end{itemize}

To reference a section by its number we can do it the same way with \texttt{\textbackslash{}@ref(label)}, e.g.:

\begin{itemize}
\tightlist
\item
  In Section \ref{sub:eventsinfty} we introduce the topic of \ldots{}
\end{itemize}

To reference a section by its title we can do this by witting the section
header in square brackets \texttt{{[}Section\ name{]}}, e.g.:

\begin{itemize}
\tightlist
\item
  In Section \protect\hyperlink{sub:eventsinfty}{Events at infinity} we introduce the topic of \ldots{}
\end{itemize}

Cross-references still work even when we refer to an item that is not on the
current page of the HTML output.

We can link to the elements from the References (Bibliography) using their keys:
\texttt{@key} gives the author and the year in brackets, use \texttt{{[}@key{]}} for both the
author and the year in brackets, e.g.:

\begin{itemize}
\tightlist
\item
  See \citet{Cohn13} for more details
\item
  See \citep{Cohn13} for more details
\end{itemize}

\backmatter

  \bibliography{references.bib}

\end{document}
